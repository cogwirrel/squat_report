\subsection{The Problem}

During training, amateur lifters will not always have access to someone that can examine their lifting technique. In practice lifters will often train with a partner, but that partner is often needed to ‘spot’ - on standby to assist the lifter should they fail to execute a lift. This means that when training it is hard to know whether you have executed a lift correctly to maximise the effectiveness of training. A lift is considered correct should it be valid in a competition setting.

It is also important that lifts are executed in a safe manner in order to minimise the risk of injury. Thus feedback on lifts is required not only to determine the validity of a lift, but also to encourage best practices in lifting technique.