\subsection{The Problem}

In a powerlifting competition, the lifts are assessed for validity by on site judges. These judges make the call as to whether a squat is deep enough or a deadlift is fully locked out. During training, amateur lifters will not always have access to a powerlifting judge. In practice lifters will often train with a partner, but that partner is often needed to ‘spot’ - on standby to assist the lifter should they fail to execute a lift. This means that when training it is hard to know whether you have executed a lift to competition standard.

It is also important that lifts are executed in a safe manner in order to minimise the risk of injury. Thus feedback on lifts is required not only to determine competition validity, but also to encourage best practices in lifting technique.

These three lifts are not only utilised in powerlifting, but have many uses in training for other sports due to their compound nature. Even individuals training to keep fit at the gym will perform these lifts to increase strength, improve endurance, keep fit and gain muscle. In these cases, it is important to be supervised while learning the lifting techniques so as to prevent injury and maximise the effectiveness of the lift.