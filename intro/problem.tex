\subsection{The Problem}

During training, amateur lifters will not always have access to someone that can examine their lifting technique. In practice lifters will often train with a partner, but that partner is often needed to `spot' (being on standby to assist the lifter should they fail to execute a lift). This means that during training it is hard for a lifter to know whether they have executed a lift correctly to maximise the effectiveness of training. A lift is considered correct should it be valid in a competition setting.

It is also important that lifts are executed in a risk-free manner in order to minimise the potential for injury. Thus feedback is required not only to determine the validity of a lift, but also to encourage best practices in performing the movement safely.