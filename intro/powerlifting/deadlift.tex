\subsection{Deadlift}

A deadlift is the final exercise performed at a powerlifting competition. In this movement, the barbell begins on the ground. The barbell is gripped in the hands of the lifter, and the lifter must successfully `stand up' whilst holding the weight. A lift is considered complete when the lifter is in a fully upright position, with knees locked and shoulders back. Figure~\ref{fig:dead_stages} shows an example of a deadlift.

\begin{figure}[H]
    \centering
    \subfigure[Begin]{
            \includegraphics[height= 7cm]{intro/images/dead_bottom}
    }
    \subfigure[End]{
            \includegraphics[height= 7cm]{intro/images/dead_top}
    }
\caption{The stages of a deadlift}
\label{fig:dead_stages}
\end{figure}

The International Powerlifting Federation regulations\cite{ipf} state that a deadlift should finish in the upright position, with no downward movement until this point. In training it is considered good practice to ensure that a `neutral spine' is maintained throughout the lift - there should be no rounding of the back and the head should stay directly in line with the lumbar.