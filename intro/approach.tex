\subsection{Approach}

The solution is to develop an application that can give feedback on a powerlifting technique, informing the user whether or not it was completed correctly and safely, and giving feedback on how the lift might be improved. It should be portable and non-intrusive, allowing an amateur lifter to take it into a gym and perform their lifts with no more equipment than they would normally use. Real-time feedback on lifts would be particularly useful, to allow the lifter to adjust their form on-the-fly. An example of this might be notifying the lifter when they have reached sufficient depth in a squat.

In order to ensure that the application is portable, the application will be developed for a mobile device. To avoid an intrusive experience, we must not rely on placing markers on the lifter - they should be able to rest their device in a suitable location while they lift, with no extra equipment about their person.

The squat is often thought to be the most important of the three powerlifting techniques in a training environment. In his book Starting Strength\cite{startingstrength} a well known strength coach, Mark Rippetoe, describes the importance of the squat in training.

\begin{quote}
\emph{There is simply no other exercise, and certainly no machine, that provides the level of central nervous activity, improved balance and coordination, skeletal loading and bone density enhancement, muscular stimulation and growth, connective tissue stress and strength, psychological demand and toughness, and overall systemic conditioning than the correctly performed full squat.}
\end{quote}

For this reason, the squat has been the primary focus of this project. The aim is create a fully functioning application to track and analyse squats in real-time on a mobile device.

The main resources of a mobile device that could be useful in tracking squats are the accelerometer and the camera. The accelerometer would provide detailed motion information when worn by the lifter, but this information would only be that of a single point. In order to effectively track and analyse squat technique, we must monitor the movement of several important points on the body. Unlike the accelerometer, the camera can be used to track any number of points due to its ability to offer a clear view of the entirety of the lifter. In light of this, the camera would be more effective in observing the movement of a lifter.

The overall approach will be to use computer vision techniques to separate the lifter from their surroundings, map a basic human model onto the lifter's silhouette, and to analyse the movement of the model to determine the validity and safety of the lift.
