\section{Evaluation}

In this section the application will be evaluated against its goal by both qualitative and quantitative means. The overarching goal of this project was to create a non-intrusive application that provides accurate real-time feedback on a lifter's squat form. The application is evaluated against this goal by examining each of the three key parts of the goal - its intrusiveness, speed, and accuracy.

\subsection{Intrusiveness}
The first of the three evaluation criteria is the application's intrusiveness. 

The application has no requirement for the user to wear any specialised equipment or markers in order to be tracked. The lifter simply follows their usual procedure in walking the weight out from the rack, squatting, then walking back to the rack to return the weight.

A small amount of set-up is required however, as the user must take their Android device into the gym and find a place to rest it so that it is stable with a clear view of the area in which they will be squatting. The user must also ensure that there is no movement in the background before they begin squatting, which can be difficult in a busy gym.

In summary, the intrusiveness of the application is in the set-up procedure, but there is none during the lift itself. From this it is fair to say that the application is non-intrusive, as once set-up is complete the application requires minimum interaction. The user does not even need to watch the screen as they squat due to the feedback being given verbally.

\subsection{Speed}

\subsubsection{Frame Rate}
A quantitative measure of the real-time nature of the application is its frame rate. Due to the way in which the application is implemented, with a separate thread processing frames to that which displays them on the screen, there are two independent frame rates that we can measure.

The first is that of the graphics thread, reading frames from the camera and displaying them on the screen. This frame rate remained level at a value of 15 frames per second, and provided a smooth video feed throughout the running of the application. This caps the maximum perceived frame rate to the user, as it is not possible to display frames at a faster rate.

The second is that of the processing thread. Measuring the rate at which the core algorithm provides frames to be displayed by the graphics thread. This frame rate fluctuated dramatically depending on the stage of the algorithm.

During the initial detection stage of the algorithm, the frame rate varied between 90 and 110 frames per second. This is due to the requirement of very little processing at this point.

The frame rate during the tracking and analysis stage of the algorithm is the most important, as this is where the algorithm is most resource intensive. During tracking, the frame rate was observed to vary between 22 and 28 frames per second. This is considerably faster than the limit of 15 frames per second, and shows that the algorithm developed in this project to track and analyse squats is easily capable of running in real-time.

\subsubsection{Feedback}
The application provides verbal feedback after each squat has been completed. This can be considered real-time as the user does not need to wait until they have finished their squats to receive feedback. This means that the user can adjust their technique during their set of squats according to the feedback given, allowing them to learn to squat safely and effectively.

\subsection{Accuracy}
Perhaps the most difficult requirement to measure is the accuracy of the application. Several approaches are taken in this section to attempt to evaluate the application's accuracy both qualitatively and quantitatively. Accuracy can be broken down into two categories - tracking accuracy and analysis accuracy.

\subsubsection{Tracking}

Much of the tracking accuracy is determined by the robustness of the background subtraction algorithm. The background subtraction algorithm is reasonably accurate, and in fixed lighting conditions performs very well, removing all of the surroundings and leaving the lifter's silhouette in place.

The algorithm taking the largest object in the foreground means that background movement that is isolated from the lifter does not have an effect on the subtraction. This can be seen in figure~\ref{fig:goodbackground}, as a man enters the view and the application continues to track as normal.

\begin{figure}[H]
    \centering
    \subfigure{
            \includegraphics[height=7cm]{evaluation/images/goodbackground1}
    }
    \subfigure{
            \includegraphics[height=7cm]{evaluation/images/goodbackground2}
    }
    \subfigure{
            \includegraphics[height=7cm]{evaluation/images/goodbackground3}
    }
\caption{Isolated background movement does not affect tracking}
\label{fig:goodbackground}
\end{figure}

Any movement in the background or foreground that crosses the lifter will effect the subtraction however, as the largest object becomes the combination of the lifter and the disturbance. This can be seen in figure~\ref{fig:badtracking}.

\begin{figure}[H]
    \centering
    \subfigure{
            \includegraphics[height=7cm]{evaluation/images/badtracking1}
    }
    \subfigure{
            \includegraphics[height=7cm]{evaluation/images/badtracking2}
    }
    \subfigure{
            \includegraphics[height=7cm]{evaluation/images/badtracking3}
    }
\caption{A passer by upsets the tracking}
\label{fig:badtracking}
\end{figure}

As the passer by walks into view, he is detected as foreground, and as soon as he crosses the lifter, becomes part of the largest object, pulling the model towards him. The model follows the passer by until he leaves the view, but after this the model is reasonably quick to snap back into place and continue tracking the lifter as usual.

In unstable lighting conditions, the background subtraction algorithm is prone to failure. When the lighting changes, the colour of pixels in the background will change and (as soon as the difference is greater than the threshold) will be detected as foreground. This then breaks down the tracking- wherever the model resides it achieves maximum overlap with the foreground, and so ceases to move. Perhaps the background subtraction algorithm could be improved to update it's known background over time, allowing for variations in lighting.

From the screenshots above, it can be seen that the application tracks reasonably well on the whole. The joint locations of the model do not always perfectly match the lifter, but the model follows the lifter with reasonable precision. The proportions of the model are fixed in the current implementation, which is likely the cause of the intermittent joint misalignment.

The combination of background subtraction, model and optimisation tracks the lifter well overall, and is robust to a reasonable level of movement in the surrounding area.

\subsubsection{Analysis}

As the application analyses squat form using computer vision, it is difficult to evaluate it quantitatively. It is possible however to compare the scores given by the application with scores given by an expert observing the squat, to give a semi-quantitative measure of the application's performance.

%TODO lots of screenshots and tables and stuff from saturday's squat sesh!


\subsection{User Testing Evaluation}
\label{sec:user_testing}
In addition to the evaluation against the goal, the application was also evaluated by a set of three users. These users are gym-goers of differing experiences: Harry Stevenson has been lifting weights for around three months, Greg Pye has been going to the gym for a little over a year, and Tom Wilding has been lifting for around four years.

These users took the application to the gym installed on a mobile device, and spent time using it to analyse their squat form.

They were asked to answer the following questions with a score out of 10 and a few words to gauge the success of the application:

\begin{enumerate}
    \item Was the application intuitive and easy to use?
    \item Was the application accurate in its tracking and analysis?
    \item Did the application aid in your training?
\end{enumerate}

They were also asked to provide any additional feedback from their real-world use of the application.

\subsubsection{Intuitiveness and Ease of Use}

Harry Stevenson:
\begin{quote}
\textbf{9/10} \emph{I felt that the app was very easy to use and the instructions that were given were clear enough so that I could understand it easily. The text-to-speech element helped me understand the instructions even though my phone was on the other side of the room.}
\end{quote}

Greg Pye:
\begin{quote}
\textbf{9/10} \emph{Very easy to use with verbal and text prompts and a simple interface. I think it'll take a user a couple of attempts of using the app with no prior instructions or help in terms of it needing the background to be still and you need to select which way to face. But after one or two tries I think it's a doddle to use.}
\end{quote}

\subsubsection{Accuracy in Tracking and Analysis}

Harry Stevenson:
\begin{quote}
\textbf{7/10} \emph{The app mostly tracked my body correctly, however on a few occasions the skeleton would track other objects or would move from my body. The analysis of the squat form was very accurate and provided useful feedback for my future use in the gym.}
\end{quote}

Greg Pye:
\begin{quote}
\textbf{9/10} \emph{As long as conditions are good the app recognises and tracks very well. Certainly far higher accuracy than I would ever expect from running such real-time processing on a phone and from its own camera too.}
\end{quote}

\subsubsection{Training Aid}

Harry Stevenson:
\begin{quote}
\textbf{9/10} \emph{The app was very beneficial in my personal training as it showed me where I had been going wrong on my form for the past three months. This app most likely saved me from having some sort of knee injury and so I found it thoroughly useful.}
\end{quote}

Greg Pye:
\begin{quote}
\textbf{7/10} \emph{I have pretty good squat form to begin with. But it did highlight a tendency at times to be a bit further forward than needs be. Nothing major to correct but will definitely keep an eye on that.}
\end{quote}

\subsubsection{Additional Feedback}

Harry Stevenson:
\begin{quote}
\emph{Sometimes the text-to-speech element would put off other gym-goers. It was sometimes difficult to find a place to put the phone so that it would stay still enough for proper usage, and in those places it did not quite have the correct angle to cover my whole body.}
\end{quote}

Greg Pye:
\begin{quote}
\emph{I think it's a great app - only limited by a gym being busy and so hard to get a still background, especially during peak times. It would be good if the app could record the session so you could play back. This means you can see how good the tracking actually was to come up with the rep's score and also watch your squat form too. I think that will be useful - especially for someone training on their own.}
\end{quote}

\subsubsection{Summary of User Evaluation}

From the feedback received we can evaluate the project as a success...???