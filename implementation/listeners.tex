\subsection{Events and Listeners}
\label{sec:listeners}

\subsubsection{Analysis and Scoring}

Events and listeners are utilised throughout the codebase, but their most interesting use is in the analysis stage.

Upon each iteration of the tracking algorithm, the model's joint positions and angles are examined. This per-frame examination is performed by the \texttt{ModelEventManager}. Other classes may register listener functions with the \texttt{ModelEventManager}, detailing which event it wishes to subscribe to. When a new observation is made on the model (for example we have newly observed that the lifter has gone `below parallel'), an event is triggered, calling every registered listener for that particular event.

The \texttt{ModelEventManager} tracks the state of the model at a low level, providing a high level view to other classes so that they can be notified when the lifter is in a particular position.

There are many events that can be subscribed to, and some of the most interesting ones are listed below:

\begin{itemize}
	\item \texttt{SQUAT\_DESCEND\_START} Triggered as soon as the lifter is detected to be descending.
	\item \texttt{SQUAT\_ASCEND\_START} Triggered as soon as the lifter is detected to be ascending.
	\item \texttt{SQUAT\_BELOW\_PARALLEL\_START} Triggered as soon as the lifter goes low enough.
	\item \texttt{SQUAT\_BELOW\_PARALLEL\_END} Triggered as soon as the lifter travels upwards out of the below parallel state.
\end{itemize}

There are many other events (listed in the appendix) that are triggered when the lifter first enters and exits a position of bad form.

This event and listener based design pattern allows a high level way to track the lifter's position. For example one could write a simple class to count the number of repetitions a lifter performs, with no need to worry about checking each frame. This example might look like this:

\begin{lstlisting}[style=javastyle]

public class SimpleRepCounter {
	private int reps = 0;

	public SimpleRepCounter(ModelEventManager manager) {
		manager.addListener(ModelEventType.SQUAT_DESCEND_START,
			new ModelEventListener() {
				public void onEvent(Model m) {
					reps++;
				}
			});
	}

	public int getReps() {
		return reps;
	}
}

\end{lstlisting}

The more advanced components used in the analysis and scoring of each repetition use sets of listeners in a similar manner.

The \texttt{ModelEventManager} provides a simple way to add new analysis functionality, whether at the low or high level.

\subsubsection{Other Listeners}

The remainder of the codebase uses listeners in order to update the user interface or provide the user with real-time feedback. This pattern was used to ensure that the core algorithm was separated from the user interface. The user interface registers itself as a listener to the \texttt{SquatPipeline}, which calls different listener functions at different stages of the algorithm. Example uses include updating the notification text in the user interface or emitting a beep whenever the lifter squats low enough.