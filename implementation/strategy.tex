\subsection{Development Strategy}

To reduce the learning curve and improve the development process, the core algorithm was first developed for a Linux machine, rather than implementing it as an Android application from the beginning. This allowed a proof of concept to be created early on in the development stages, reading and processing a video file rather than live images from the camera, giving a more controlled atmosphere for testing. Having never developed an Android application, or used OpenCV, developing on Linux first allowed me to familiarise myself with OpenCV development without concern for any Android-specific functionality. With the Linux code all written in Java, the overhead for porting the Linux implementation to Android was minimal.

Platform-specific functionality was abstracted using interfaces during the Linux development process. Once the core algorithm had been completed, the migration from Linux to Android was reasonably straightforward. The Linux implementation of the \texttt{VideoInput} interface (which read frames from a video file) needed to be replaced with an Android implementation (reading frames from the camera). The same needed to be done for the \texttt{VideoDisplay} interface. The OpenCV library for Linux was swapped for its equivalent on Android. Once these steps were taken, the core algorithm ran just as it had done on Linux, which enabled the focus to shift from the algorithm to user interaction.