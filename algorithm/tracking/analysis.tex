\subsubsection{Analysis}

The final stage of the algorithm is the analysis. This component of the algorithm examines the angles and positions of the model's joints, to provide information as to whether the model is currently in a good or bad position.

The model provides methods to query its state. An example might be a method that returns whether the hip joint is below the knee joint, allowing us to determine whether the lifter is currently `below parallel' - in the position that they must reach at the bottom of a squat in order for it to be valid. 

On every frame, the model is queried for it's current position, and a global `picture' of the current state of the squat is built up. Exactly how this picture is built up is detailed in section~\ref{sec:listeners}.

The analysis tracks joint positions over time to enable one squat repetition to be differentiated from another. A particular challenge was in how to determine when the lifter is ascending or descending. A simple solution would be to check whether the lifter has just locked out (stood up straight) or has gone below parallel. However in practice a lifter may not perform a correct squat, and may neither squat below parallel nor lock out. The algorithm settled upon was to track the hip joint's vertical motion. As the lifter descends, the hip joint will descend, and as the lifter ascends, so will the hip joint. A fixed size queue of hip joint locations is stored, and on every frame the location is added to the queue (pushing off the oldest hip joint location). This gives a short `history' of hip locations, and we can examine this history to see whether the lifter is currently ascending or descending by checking whether the history is in ascending or descending order. This provides a way to tell when the lifter is ascending and descending, and after each descension and ascension we know that a new repetition has begun.

In the analysis stage, we give each repetition a percentage score, which is calculated by penalising failure to enter required positions (for example going deep enough and standing up straight enough), and also penalising time spent in a sub-optimal or unsafe position (such as leaning too far forward). The time spent is measured in frames, and percentage points are removed from the score based on the proportion of frames in which the lifter is an incorrect position. Each type of incorrect position is given a maximum penalty at which no more percentage points are removed. The maximum penalties chosen are based on the severity of the incorrect position.

As we score each squat, we keep track of the most significant contributor to the score penalty. For example, if 10\% of the score was lost for leaning too far backward, and a 30\% penalty was applied for the lifter's failure to squat low enough, the main contributor would be the fact that the lifter failed to squat low enough. These contributors are stored with each score, enabling the contributor to be shown to the user to allow them to see what the main problem was with their squat, so that they might know how to improve their technique.