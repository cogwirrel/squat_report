\subsection{Analysis}

The final stage of the algorithm is the analysis. This component examines the angles and locations of the model's joints to provide information as to whether the model is currently in a good or bad position.

The model provides methods to query its state. On every frame the model is queried for its current position and analysis components are notified of changes in the model's state. This is detailed further in section~\ref{sec:listeners}.

The analysis tracks joint positions over time to enable one squat repetition to be differentiated from another. A particular challenge was how to determine when the lifter is ascending or descending. A simple solution would be to check whether the lifter has just locked out (stood up straight) or has gone below parallel. However in practice a lifter may not perform a correct squat, and may neither squat below parallel nor lock out. The algorithm settled upon was to track the hip joint's vertical motion. As the lifter descends, the hip joint will descend, and as the lifter ascends, so will the hip joint. A fixed-size queue of hip joint locations is stored and on every frame the new location is added to the queue (discarding the oldest hip joint location). This gives a short `history' of hip locations and this history can be examined to see whether the lifter is currently ascending or descending by checking whether the history is in ascending or descending order. Upon each new descent it is inferred that a new repetition has begun.

In the analysis stage, each repetition is given a percentage score. This is calculated by penalising failure to enter required positions (for example going deep enough and standing up straight enough) and by penalising time spent in a sub-optimal or unsafe position (such as leaning too far forward). The time spent is measured in frames, and percentage points are removed from the score based on the proportion of frames in which the lifter is an incorrect position. Each type of incorrect position is given a maximum penalty at which no more percentage points are removed. The maximum penalties have been chosen based on the severity of the incorrect position.

As each squat is scored, the most significant contributor to the score penalty is tracked. For example, if 10\% of the score was lost for leaning too far backward, and a 30\% penalty was applied for the lifter's failure to squat low enough, the main contributor would be the fact that the lifter failed to squat low enough. These contributors are stored with each score, enabling the contributor to be shown to the user to allow them to see what the main problem was with their squat. These contributors are used after each repetition to provide verbal feedback on how to improve their technique.