\section{Background}

%Background (10- 20 pages).  This should form the bulk of the interim report. You should consider that your objective here is to produce a near final version of the background section, as it will appear in your final report.  All of this material should be re-usable, so it is worth getting it right at this stage of the project.  The details of what to include can be found in the Project Report guidelines.

%The background section of the report should set the project into context by relating it to existing published work which you read at the start of the project when your approach and methods were being considered. There are usually many ways of solving a given problem, and you shouldn't just pick one at random. Describe and evaluate as many alternative approaches as possible. The published work may be in the form of research papers, articles, text books, technical manuals, or even existing software or hardware of which you have had hands-on experience. Your must acknowledge the sources of your inspiration. You are expected to have seen and thought about other people's ideas; your contribution will be putting them into practice in some other context. However, avoid plagiarism: if you take another person's work as your own and do not cite your sources of information/inspiration you are being dishonest; in other words you are cheating. When referring to other pieces of work, cite the sources where they are referred to or used, rather than just listing them at the end. Make sure you read and digest the Department's plagiarism document .

%In writing the Background chapter you must demonstrate your capability of analysis, synthesis and critical judgement. Analysis is shown by explaining how the proposed solution operates in your own words as well as its benefits and consequences. Synthesis is shown through the organisation of your Related Work section and through identifying and generalising common aspects across different solutions. Critical judgement is shown by discussing the limitations of the solutions proposed both in terms of their disadvantages and limits of applicability.

\subsection{A Stereo Approach}

A large amount of research has been done in the use of stereo cameras and 3D sensors, for skeletal tracking. Much of this arose from the introduction of Microsoft's Kinect\cite{kinect} sensor which brought affordable and reliable depth perception to the masses. Library's such as OpenNI\cite{openni} provide methods for skeletal tracking with a 3D sensor straight out of the box.

A company known as Captury\cite{captury} have constructed a method for robust motion capture using a stereo camera. In their paper `On-set Performance Capture of Multiple Actors With A Stereo Camera'\cite{capturystereopaper} they describe a method that ``employs appearance cues, scene flow, pose reconstruction results from previous frames, and stereo coherence to reliably segment out actors in front of general backgrounds.''. This method has been shown to be highly robust, and Captury have provided a video\cite{capturyvideo} to demonstrate this. Not only does their method track the skeleton, it also performs a further step to map a full three dimensional body to the video, for full motion capture.

Although immensely accurate, Captury's process requires a large amount of computation, in the order of minutes per frame. This means that use of an algorithm following their pipeline would not be able to achieve real time feedback.

Unfortunately, despite the wide array of tools and libraries available for skeletal tracking with a 3D sensor or stereo camera, it would be infeasible to take a device like a Kinect hooked up to a computer into the gym. Perhaps if the goal were to have the system set up at powerlifting competitions, or pre-installed at gyms, this method would be more reasonable. However as a personal training aid, the device must be portable.

Another alternative option would be to develop for a smartphone with a stereo camera. Smartphones such as the LG Optimus 3D\cite{lgoptimus} provide such a camera. However these phones are not currently in wide use and would severely narrow the audience for this application.

