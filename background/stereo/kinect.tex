\subsubsection{Kinect}

Jamie Shotton et al developed an algorithm\cite{shottonkinect} for human pose detection (that is the three-dimensional positions of body joints) which uses a Randomised Decision Forest trained on an extremely large data set of a wide variety of poses. This machine learning approach is used in the Kinect tracking system, and has the advantage of being both robust and computationally inexpensive to classify a new image. This is due to the algorithm reducing the detection into a per-pixel classification problem. Another advantage of this method is that it does not rely on temporal information (or previous frames) as pose estimation can be performed on a standalone frame.

However this machine learning approach may not be feasible in the scope of this project. There is not enough time available to collect the data required to train a machine learner such as this to recognise body pose. Shotton's results showed that in order to effectively detect a human pose, his system required training on at least 100,000 images\cite{shottonkinect}.