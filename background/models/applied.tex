\subsubsection{Applied to this Project}

As this project will be using a monocular camera, the mapping of a three-dimensional model to a two-dimensional frame is more computationally challenging than if a depth camera was used. In the case of squatting, all required data to determine the validity of a lift can be extracted from a two-dimensional model of the side view. The only information that would be missed with a two-dimensional model is one related to the safety of the lifts: the position of the knees from a front view, to make sure that they do not buckle in towards each other. This is an acceptable compromise, being only a single measure out of many and is more easily observed by the lifter if squatting in front of a mirror.

The camera will be in a known location relative to the human subject and the movements are well defined, which makes a two-dimensional model feasible. As it will provide enough data to determine the validity of a lift and to measure the majority of its safety criteria, a two-dimensional model will be most practical for this project.

This two-dimensional model will likely be based on the cardboard model, using two-dimensional shapes that best fit the human subject. Their positions should not individually be defined by variables such as pitch and yaw, but by the rotation of each joint. This will reduce the number of degrees of freedom of the model, allowing for faster optimisation. The natural limits of human joint movement can be used to constrain the optimisation problem.