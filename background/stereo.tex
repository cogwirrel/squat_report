\subsection{A Stereo Approach}

A large amount of research has been done in the use of stereo cameras and 3D sensors, for skeletal tracking. Much of this arose from the introduction of Microsoft's Kinect\cite{kinect} sensor which brought affordable and reliable depth perception to the masses. Library's such as OpenNI\cite{openni} provide methods for skeletal tracking with a 3D sensor straight out of the box.

A company known as Captury\cite{captury} have constructed a method for robust motion capture using a stereo camera. In their paper `On-set Performance Capture of Multiple Actors With A Stereo Camera'\cite{capturystereopaper} they describe a method that ``employs appearance cues, scene flow, pose reconstruction results from previous frames, and stereo coherence to reliably segment out actors in front of general backgrounds.''. This method has been shown to be highly robust, and Captury have provided a video\cite{capturyvideo} to demonstrate this. Not only does their method track the skeleton, it also performs a further step to map a full three dimensional body to the video, for full motion capture.

Although immensely accurate, Captury's process requires a large amount of computation, in the order of minutes per frame. This means that use of an algorithm following their pipeline would not be able to achieve real time feedback.

Unfortunately, despite the wide array of tools and libraries available for skeletal tracking with a 3D sensor or stereo camera, it would be infeasible to take a device like a Kinect hooked up to a computer into the gym. Perhaps if the goal were to have the system set up at powerlifting competitions, or pre-installed at gyms, this method would be more reasonable. However as a personal training aid, the device must be portable.

Another alternative option would be to develop for a smartphone with a stereo camera. Smartphones such as the LG Optimus 3D\cite{lgoptimus} provide such a camera. However these phones are not currently in wide use and would severely narrow the audience for this application.

