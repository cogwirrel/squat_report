\subsubsection{Background Subtraction}

This method requires a background frame clear of anything in the foreground, to compare it to subsequent frames containing one or more foreground objects. The per pixel difference between the known background frame and subsequent foreground frames is computed. Differences above a certain threshold are considered foreground, and differences below this threshold are considered background.

The background image can also be obtained by taking the average pixel values over a selection of frames in a video sequence.

More advanced implementations of background subtraction will use mixtures of gaussians to represent colours in the frame, with each gaussian weighted by the time that colour remains in the scene\cite{backgroundsubmog}. Colours considered to be background are those that stay in the scene longer, with less motion. Implementations such as this are used in OpenCV\cite{opencv}.