\subsection{Background Subtraction}

This method requires a background frame clear of anything in the foreground. This background frame is compared with each subsequent frame by calculating the per pixel difference. Pixels with a difference above a certain threshold are considered foreground, and those below this threshold are considered background.

The background image can also be obtained by taking the average pixel values over a selection of frames in a video sequence, resulting in a more adaptive algorithm.

More advanced implementations of background subtraction will use mixtures of gaussians to represent colours in the frame, with each gaussian weighted by the time that colour remains in the scene\cite{backgroundsubmog}. Colours considered to be background are those that stay in the scene longer, with less motion. Implementations such as this are provided by OpenCV\cite{opencv}.