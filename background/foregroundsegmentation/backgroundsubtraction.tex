\subsubsection{Background Subtraction}

The background subtraction in general is to compute the per pixel differences between a given frame and a known background frame. Differences above a certain threshold are considered foreground, and differences below this threshold are considered background.

The background image is often obtained from either a frame from the video sequence known to be background, or by taking the average pixel values over a selection (or all) frames in a video sequence.

More advanced implementations of background subtraction will use mixtures of gaussians to represent colours in the frame, with each gaussian weighted by the time that colour remains in the scene\cite{backgroundsubmog}. Colours considered to be background are those that stay in the scene longer, with less motion. Implementations such as this are used in OpenCV\cite{opencv}.